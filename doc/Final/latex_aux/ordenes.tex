\usepackage {ifthen}
\usepackage{colortbl}

% ------------------ Macro para insertar una imagen ---------------------------
%       Uso: \imagen{nombreFichero}{Ancho}{Etiqueta}{Identificador}
% -----------------------------------------------------------------------------
\def\imagen#1#2#3#4{
 \begin{figure}[here]
 \begin{center}
   \resizebox{#2}{!}{\includegraphics{#1}}
 \ifthenelse{\equal{#3}{}}{}{\caption {#3}}
 \label{'#4'}
 \end{center}
 \end{figure}
}

% ------------------ Macro para insertar un panel en gris ---------------------
%       Uso: \panel[Opcional: Ancho del panel en tanto por uno]
% -----------------------------------------------------------------------------
\newenvironment{panel}[1][0.9]{%
  \setlength\arrayrulewidth{1pt}
  \begin{tabular}{|>{\columncolor[gray]{.9}}l|}
    \hline
    \begin{minipage}{#1\textwidth}\par\rule{0cm}{1mm}\par}{%
      \vspace{3mm}\end{minipage}\\
    \hline
  \end{tabular}
}

% ------------------ Macro para cambiar el interlineado ---------------------
%       Uso: \interlineado{tama�o en puntos: 1.0 es normal, 2.0 doble...}
% -----------------------------------------------------------------------------
\newcommand{\interlineado}[1]{
    \renewcommand{\baselinestretch}{#1}  % -- Cambiamos interlineado
	 \large\normalsize % ---------------------- Para que cambie de verdad
}
