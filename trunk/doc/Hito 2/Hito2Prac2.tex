\documentclass[a4paper,12pt,titlepage]{article}
\usepackage[T1]{fontenc}
\usepackage[spanish]{babel}
\usepackage[latin1]{inputenc} %% Soporte tildes, � (NO comillas) y separacion palabras
\usepackage{listings}
\usepackage{color}
\usepackage{fancyhdr}
\usepackage{graphicx}


% Comandos para definir lineas del encabezado y pie de pagina
\renewcommand{\headrule}{\hrule width\headwidth height\headrulewidth \vskip-\headrulewidth}
\renewcommand{\footrulewidth}{0.4pt}


% M�rgenes del documento
\topmargin=-0.5cm
\textwidth=15cm
\textheight=22.5cm
\oddsidemargin=0.5cm
 
 
\setlength{\pdfpagewidth}{\paperwidth}
\setlength{\pdfpageheight}{\paperheight}
 
% Definicion del "listings" para el lenguaje python
\lstdefinelanguage{python}
{
 morecomment = [l]{\#}, 
 morestring=[b]", 
 sensitive = true,
 morekeywords = {self, print, from, import, if, True, else, elif, False, while, for, min, max, or, and, class, def, open, close, read, readlines, write, len, range, append, return}
}


%%Portada
\pagestyle{fancy}
\title{\textrm{\textbf{\textsc{Inteligencia Artificial}}}\\
\textsf{Segunda Pr�ctica. Hito 2: Algoritmo Minimax y poda alfa-beta}}
\author{Jose Domingo L�pez L�pez\\}
 
%% Encabezado y pie de p�gina
\fancyhead[L]{\sc \small B�squeda entre adversarios. Hito 2}
\fancyhead[R]{}
\fancyfoot[L]{\small Jose Domingo L�pez L�pez}
\fancyfoot[R]{\thepage}
\fancyfoot[C]{}


% Configuracion de la apariencia del codigo incluido
\lstset{ %
  language=python,
  basicstyle=\scriptsize,
  % numbers=left,
  % numberstyle=\footnotesize,
  % stepnumber=2,
  % numbersep=5pt,
  backgroundcolor=\color{white},
  showspaces=false,
  showstringspaces=false,
  showtabs=false,
  tabsize=2,   
  captionpos=b,
  breaklines=true,
  breakatwhitespace=false,
  escapeinside={\#},
  keywordstyle=\color[rgb]{0,0,1},
  commentstyle=\color[rgb]{0.133,0.545,0.133},
  stringstyle=\color[rgb]{0.627,0.126,0.941}
}

\begin{document}

%% Imagen de la portada
\begin{figure}[t]
\centering
\includegraphics[width=6cm]{Ferrita}
\end{figure}

\maketitle

\thispagestyle{empty}
\tableofcontents
\newpage

\section{Clase Casilla}
\texttt{\lstinputlisting[inputencoding=latin1]{../src/casilla.py}}

\newpage
\section{Clase Cliente}
\texttt{\lstinputlisting[inputencoding=latin1]{../src/client.py}}

\newpage
\section{Clase Equipo}
\texttt{\lstinputlisting[inputencoding=latin1]{../src/equipo.py}}

\newpage
\section{Clase Estado}
\texttt{\lstinputlisting[inputencoding=latin1]{../src/estado.py}}

\newpage
\section{Clase Jugador}
\texttt{\lstinputlisting[inputencoding=latin1]{../src/jugador.py}}

\newpage
\section{Clase Minimax}
\texttt{\lstinputlisting[inputencoding=latin1]{../src/minimax.py}}

\newpage
\section{Clase Nodo}
\texttt{\lstinputlisting[inputencoding=latin1]{../src/nodo.py}}

\newpage
\section{Clase Tablero}
\texttt{\lstinputlisting[inputencoding=latin1]{../src/tablero.py}}

\newpage
\section{Constantes}
\texttt{\lstinputlisting[inputencoding=latin1]{../src/config.py}}

\newpage
\section{Variables globales}
\texttt{\lstinputlisting[inputencoding=latin1]{../src/global_vars.py}}

\end{document}