% -*- coding: utf-8 -*-
\documentclass [a4paper, oneside, 12pt]{book}
% \documentclass[a4paper,12pt]{article}
\usepackage[spanish]{babel}
\usepackage {latexsym}
\usepackage {fancybox}
\usepackage {longtable}
\usepackage {hyperref}
\hypersetup{linktocpage,colorlinks,urlcolor=blue,citecolor=blue,%
	linkcolor=blue,filecolor=blue,pdfnewwindow}
\usepackage {a4wide}

\usepackage {color}
\usepackage[utf8]{inputenc}
\ifx\pdftexversion\undefined
  \usepackage[dvips]{graphics}
\else
  \usepackage[pdftex]{graphics}
\fi

%% margenes, tamaño del texto, etc
\setlength{\headheight}{17pt} %
\setlength{\oddsidemargin}{9pt} %
\setlength{\topmargin}{0pt} %
\setlength{\textheight}{650pt} %
\setlength{\textwidth}{445pt} %
\setlength{\marginparwidth}{60pt} %

\usepackage[nottoc]{tocbibind}

%Para encabezados y pies de página
\usepackage{fancyhdr}

% Para insertar código fuente desde archivos
\usepackage{listings}

% Definicion del "listings" para el lenguaje python
\lstdefinelanguage{python}
{
 morecomment = [l]{\#}, 
 morestring=[b]", 
 sensitive = true,
 morekeywords = {self, print, from, import, if, True, else, elif,
   False, while, for, min, max, or, and, class, def, open, close,
   read, readlines, write, len, range, append, return, function}
}

% Configuracion de la apariencia del codigo incluido
\lstset{ %
  language=python,
  basicstyle=\scriptsize,
  % numbers=left,
  % numberstyle=\footnotesize,
  % stepnumber=2,
  % numbersep=5pt,
  backgroundcolor=\color{white},
  showspaces=false,
  showstringspaces=false,
  showtabs=false,
  frame=single,
  tabsize=2,
  captionpos=b,
  breaklines=true,
  breakatwhitespace=false,
  escapeinside={\#},
  keywordstyle=\color[rgb]{0,0,1},
  commentstyle=\color[rgb]{0.133,0.545,0.133},
  stringstyle=\color[rgb]{0.627,0.126,0.941}
}


\headsep=8mm
\footskip=14mm

%%% Fancy Header %%%%%%%%%%%%%%%%%%%%%%%%%%%%%%%%%%%%%%%%%%%%%%%%%%%%%%%%%%%%%%%%%%
% Fancy Header Style Options
\pagestyle{fancy}                       % Sets fancy header and footer
\fancyfoot{}                            % Delete current footer settings
\renewcommand{\chaptermark}[1]{         % Lower Case Chapter marker style
   \markboth{\chaptername\ \thechapter.\ #1}{}} %
\renewcommand{\sectionmark}[1]{         % Lower case Section marker style
   \markright{\thesection.\ #1}}         %
   \fancyhead[LE,RO]{\bfseries\thepage}    % Page number (boldface) in left on even
                                           % pages and right on odd pages
   \fancyhead[RE]{\bfseries\leftmark}      % Chapter in the right on even pages
   \fancyhead[LO]{\bfseries\rightmark}     % Section in the left on odd pages
   \renewcommand{\headrulewidth}{0.5pt}    % Width of head rule

%Para los mini-índices de cada capítulo
\usepackage{minitoc}
\setcounter{minitocdepth}{2}         %defecto es 2
\setlength{\mtcindent}{0pt}          %defecto es 24
\renewcommand{\mtcfont}{\small\rm}   %defecto
\renewcommand{\mtcSfont}{\small\bf}  %defecto
\renewcommand{\mtcSSfont}{\small\rm} %defecto es \small\bf
\renewcommand{\mtctitle}{En este cap�tulo...}
\nomtcpagenumbers %no poner número de página en mini-índice
% ------------------------------------------------------------------------------
%Para la tabla de contenido general
\setcounter{tocdepth}{2}
\setcounter{secnumdepth}{3}
% ------------------- Opciones específicas de este documento -----------------
\usepackage{times}  % bookman | newcent | palatino | times

\graphicspath{{imagenes/}}
\usepackage {ifthen}
\usepackage{colortbl}

% ------------------ Macro para insertar una imagen ---------------------------
%       Uso: \imagen{nombreFichero}{Ancho}{Etiqueta}{Identificador}
% -----------------------------------------------------------------------------
\def\imagen#1#2#3#4{
 \begin{figure}[here]
 \begin{center}
   \resizebox{#2}{!}{\includegraphics{#1}}
 \ifthenelse{\equal{#3}{}}{}{\caption {#3}}
 \label{'#4'}
 \end{center}
 \end{figure}
}

% ------------------ Macro para insertar un panel en gris ---------------------
%       Uso: \panel[Opcional: Ancho del panel en tanto por uno]
% -----------------------------------------------------------------------------
\newenvironment{panel}[1][0.9]{%
  \setlength\arrayrulewidth{1pt}
  \begin{tabular}{|>{\columncolor[gray]{.9}}l|}
    \hline
    \begin{minipage}{#1\textwidth}\par\rule{0cm}{1mm}\par}{%
      \vspace{3mm}\end{minipage}\\
    \hline
  \end{tabular}
}

% ------------------ Macro para cambiar el interlineado ---------------------
%       Uso: \interlineado{tama�o en puntos: 1.0 es normal, 2.0 doble...}
% -----------------------------------------------------------------------------
\newcommand{\interlineado}[1]{
    \renewcommand{\baselinestretch}{#1}  % -- Cambiamos interlineado
	 \large\normalsize % ---------------------- Para que cambie de verdad
}


% ----------------- Definimos el interlineado Base para todo el documento ------
\newcommand{\interlineadoBase}{1.5}  
%%%%%%%%%%%%%%%%%%%%%%%%%%%%%%%%%%%%%%%%%%%%%%%%%%%%%%%%%%%%%%%%%%%%%%%%%%%%%%%%

\interlineado{1.0}
% -------------------------------------------  Creamos tablas de contenido -----
\dominitoc[e] %el parámetro indica la posición del título. r=right, n=e=empty

% -----[x]------------- Orden auxiliar para escribir la minitoc ----------------
\def\escribeMiniToc{
    \interlineado{1.0}
    \minitoc
    \interlineado{\interlineadoBase}
}

\begin {document}

  \thispagestyle{empty}
  \vspace*{-2cm}
\begin{center}
  \vspace*{1.5cm}
    \interlineado{2.0}
    {\LARGE \textbf{UNIVERSIDAD DE CASTILLA-LA MANCHA\\ 
        ESCUELA SUPERIOR DE INFORMÁTICA}}

    \imagen{Ferrita}{6cm}{}{FIGlogouclm}

    {\huge \textbf{INGENIERÍA EN INFORMÁTICA}}

    \vspace*{1.5cm}
    {\LARGE \textbf{Inteligencia Artificial e Ingeniería del Conocimiento}}

    \vspace*{1.5cm}
    {\huge Práctica 2: Búsqueda entre adversarios}
    \interlineado{\interlineadoBase}
  \vfill
  \hfill
\begin{flushright}
  {\large Jose Domingo López López\\josed.lopez1@alu.uclm.es}

  \vspace* {1.0cm}
  {\large \the\day{} de \ifcase\month\or Enero\or Febrero\or Marzo\or Abril\or Mayo\or%
    Junio\or Julio\or Agosto\or Septiembre\or Octubre\or Noviembre\or
    Diciembre\fi{} de \the\year}
\end{flushright}
\end{center}
                                % segunda pagina
  \newpage
\chapter*{}
\thispagestyle{empty}  % Suprime la numeraci�n de esta p�gina
\vspace{16cm}
\begin{small}
\copyright~ Jose Domingo López López. Se permite la copia, distribución y/o
modificación de este documento bajo los términos de la licencia de
documentación libre GNU, versión 1.1 o cualquier versión posterior publicada
por la {\em Free Software Foundation}, sin secciones invariantes. Puede
consultar esta licencia en http://www.gnu.org. \\[0.2cm]
Este documento fue compuesto con \LaTeX{}. Imágenes generadas con The GIMP.
\end{small}

\pagenumbering{Roman}  % --------------------- Numeraci�n en romanos -----------

\interlineado{1.0}
\tableofcontents
\mtcaddchapter
% \listoffigures
% \mtcaddchapter
% \listoftables

% --------------------------------------- Introducci�n y Planteamiento ---------
% ------------------------------------------------------------------------------

\interlineado{\interlineadoBase}

%\chapter{Introducción}
%\escribeMiniToc

\pagenumbering{arabic}  % --------------------- Numeraci�n en ar�bigos -----------
\setcounter{page}{1}

% Empezar el fichero con:
% \section{Plantilla para el PFC}
% y seguir con las correspondientes sub...section
% \label {sec:etiqueta_identificativa}
% \section{putas}
\subsection{putanumerone}
\label {sec:introduccion}
% -*- coding: utf-8 -*-

\chapter {Introducción}
Para comenzar con este documento se explicará en qué consiste el
problema que abordaremos y cuáles son las decisiones de diseño
planteadas, así como las estructuras de datos que se utilizarán para la
resolución del mismo.


\section {Definición del problema}
El problema consiste en realizar un bot (agente racional) que pueda
jugar de forma autónoma contra un adversario en \textbf{El Juego de la
Bandera}, permitiendo la posibilidad de realizar una pequeña
competición. La estrategia de juego será dirigida por un algoritmo
Mini-Max con poda alfa-beta.

Se puede encontrar una especificación más detallada de en qué consiste
\textbf{El Juego de la Bandera} en el enunciado de la primera práctica
de \emph{Inteligencia Artificial e Ingeniería del Conocimiento} del
curso académico 2008-2009.


\section {Decisiones de diseño}
Al abordar este problema se plantearon una serie de conflictos
(distancias mínimas reales, heurísticas, etc) cuya resolución no fue
trivial y, dada su complejidad, se explicarán en apartados posteriores
dedicados expresamente a ellos. No obstante, a este nivel ya se puede
hablar de tres decisiones de diseño de gran importancia:
\begin{itemize}
\item El temporizador.
\item El método de búsqueda.
\item Control de estados repetidos.
\end{itemize}

\subsection{Temporizador}
Como se define en el enunciado del problema, el bot debe jugar de
forma autónoma contra un adversario y existe un parámetro fijado al
crear una partida que indica el tiempo que tiene cada jugador para
``pensar'' su jugada. Este parámetro acotará la profundidad a la que
puede llegar el algoritmo Mini-Max y es crucial mandar un movimiento
al servidor dentro de este límite de tiempo.

Este problema se resolverá por medio de \textbf{hilos}. El programa
principal pondrá a ``falso'' una variable compartida que hace la
función de semáforo y creará un hilo. El hilo lanzará el algoritmo de
búsqueda y el programa principal dormirá una cantidad de tiempo cercana a la
duración del turno. Al despertar, pondrá a ``verdadero'' el semáforo y
el hilo que lanzó el algoritmo de búsqueda parará su ejecución. Para
que todo esto funcione correctamente, el algoritmo de búsqueda debe ir
guardando en todo momento el mejor nodo que ha ido encontrando.

\subsection{Método de búsqueda}
Generalmente los algoritmos Mini-Max hacen una búsqueda clásica en
profundidad, pero cuando el árbol de búsqueda es muy extenso y estamos
limitados por tiempo, este método puede hacerse inviable ya que
podemos dejar sin explorar zonas del árbol muy prometedoras. Para una
búsqueda más exhaustiva, emplearemos un método de búsqueda de
\textbf{profundidad iterativa}. De este modo nos aseguramos el ir
explorando cada nivel por completo y nos es posible guardar la mejor
solución encontrada hasta el momento.

\subsection{Control de estados repetidos}
\label{sec:estadosrepetidos}
Como se verá a lo largo de este documento, el algoritmo Mini-Max tiene
una función muy importante llamada \emph{expandir} que se encarga de
generar todos los sucesores de un determinado nodo. Es posible que los
sucesores generados ya hayan sido generados en algún otro momento y
estén en espera para ser evaluados o expandidos, o incluso que ya
hayan sido evaluados y expandidos.

Para evitar este crecimiento innecesario del árbol de búsqueda se
tratará de controlar la creación de estados repetidos mediante dos
estructuras de datos:
\begin{enumerate}
\item Lista de estados repetidos.
\item Diccionario de estados repetidos, simulado una tabla hash.
\end{enumerate}

\section {Estructuras de datos}
En la figura \ref{'classdiagram'} se  muestra el diagrama de clases a
partir del cual afrontaremos la programación de la práctica. Posteriormente, se
mostrarán cada una de las clases individualmente y se dará una breve
explicació de sus métodos y atributos.

\imagen{class_diagram}{17cm}{Diagrama de clases}{classdiagram}

\subsection {Clase Casilla}
Esta clase (ver figura \ref{'classdiagramcasilla'}) almacena la
información relativa a una casilla dada.
\begin{itemize}
\item \textbf{Atributos}: identificador y tipo.
\item \textbf{Métodos}: \emph{convertirHierba()} y \emph{cavar()} únicamente
  actualizan el atributo \emph{tipo} del objeto. La operación
  \emph{coste()} devuelve un entero que indica cuánta vida la cuesta a
  un jugador moverse a dicha casilla, teniendo en cuenta las unidades
  de hacha (para bosques) y barca (para agua) de ese jugador.
\end{itemize}

\imagen{class_diagram_casilla}{5cm}{Clase
  Casilla}{classdiagramcasilla}

\subsection {Clase Tablero}
Esta clase (ver figura \ref{'classdiagramtablero'}) almacena los
cambios que han sucedido en el tablero desde el inicio hasta un
momento dado. De este modo, podemos calcular el estado actual del
tablero sin tener que almacenar información que no ha cambiado.
\begin{itemize}
\item \textbf{Atributos}: lista de casillas modificadas y número de banderas
  que hay actualmente en el tablero.
\item \textbf{Métodos}.
  \begin{itemize}
  \item \emph{idCasillasVecinas()} devuelve una lista de
    identificadores de las seis casillas vecinas a la casilla indicada.
  \item \emph{casillaActual()} devuelve un objeto Casilla dado un
    identificador de casilla. Este objeto contiene el tipo actual de
    la casilla en ese momento.
  \item \emph{casillasVecinasActuales()} devuelve una lista que
    contiene las seis casillas vecinas al identificador dado. Estos
    objetos Casilla contienen su tipo actual.
  \item \emph{WayTracking()} es un algoritmo que calcula las
    distancias reales desde cualquier casilla del tablero a una
    casilla dada. Más adelante se dará una descripción más detallada
    de este algoritmo.
  \end{itemize}
\end{itemize}

\imagen{class_diagram_tablero}{9cm}{Clase
  Tablero}{classdiagramtablero}

\subsection {Clase Jugador}
Esta clase (ver figura \ref{'classdiagramjugador'}) almacena la
información relativa a cada jugador tal y como se indica en el
enunciado de la práctica.
\begin{itemize}
\item \textbf{Atributos}: identificador del jugador, identificador del
  equipo, casilla actual, energia y las unidades de los objetos que posee.
\item \textbf{Métodos}: Como se puede observar, prácticamente todos
  los métodos devuelven un \emph{boolean}. Se trata de los métodos que
  implican coger un objeto o utilizarlo, por lo que esta variable booleana
  indica si el jugador ha cogido un objeto o lo ha utilizado. La
  importancia de esta decisión de diseño radica en poder saber si el
  jugador se ha movido o no, de este modo únicamente se generarán los
  sucesores que han generado algún cambio en el estado de la partida.
\end{itemize}

\imagen{class_diagram_jugador}{5cm}{Clase Jugador}{classdiagramjugador}

\subsection {Clase Equipo}
Esta clase (ver figura \ref{'classdiagramequipo'}) almacena la
información relativa a un equipo. En nuestro problema únicamente
existirán dos instancias de esta clase: una para el equipo MIN y otra
para el equipo MAX.
\begin{itemize}
\item \textbf{Atributos}: identificador del equipo, banderas
  capturadas por el equipo y la lista de jugadores que pertenecen al equipo.
\item \textbf{Métodos}: \emph{getEnergiaTotal()} suma las energias de
  todos los jugadores del equipo. Este método es importante para la
  heurística que valore lo buena o mala que sea la cantidad de energía
  de un equipo.
\end{itemize}

\imagen{class_diagram_equipo}{5cm}{Clase Equipo}{classdiagramequipo}

\subsection {Clase Estado}
Esta clase (ver figura \ref{'classdiagramestado'}) almacena la
información relevante de un estado.
\begin{itemize}
\item \textbf{Atributos}: una instancia de la clase Tablero que indica
  las modificaciones que se han hecho en el tablero desde el inicio
  hasta este estado, y la lista de equipos que juegan la partida con
  el estado de cada uno de sus jugadores.
\item \textbf{Métodos}.
  \begin{itemize}
  \item \emph{esSolucion()} devuelve ``verdadero'' si el número de
    banderas que quedan en el tablero es 0. En caso contrario devuelve
    ``falso''.
  \item \emph{actualizarEstado()} actualiza la información del jugador con
    el que se realiza una acción, la información de su equipo y la
    información de la casilla a la que se mueve. Devuelve
    ``verdadero'' en caso de que se haya modificado el estado y
    ``falso'' en caso contrario. Esta decisión es importante para
    únicamente generar nodos que tengan modificaciones en su estado.
  \end{itemize}

\end{itemize}

\imagen{class_diagram_estado}{6cm}{Clase Estado}{classdiagramestado}

\subsection {Clase Nodo}
Esta clase (ver figura \ref{'classdiagramnodo'}) almacena la
información perteneciente a cada nodo. Esta información está reflejada
en sus atributos.
\begin{itemize}
\item \textbf{Atributos}: el estado de este nodo, el nodo padre, la
  acción que se ha ejecutado para llegar a este nodo, la profundidad
  del nodo y su valor de utilidad (inicialmente valorado a -INFINITO).
\item \textbf{Métodos}.
  \begin{itemize}
  \item \emph{repetidoEnRama()} que comprueba si este nodo contiene un
    estado repetido en alguno de sus antecesores.
  \item \emph{primerAntecesor()} que devuelve el nodo con profundidad
    inicial que nos ha llevado a generar este nodo.
  \end{itemize}
\end{itemize}

\imagen{class_diagram_nodo}{5cm}{Clase Nodo}{classdiagramnodo}

\subsection {Clase Minimax}
Esta clase (ver figura \ref{'classdiagramminimax'}) es la clase
principal que maniobra con todas las estructuras en busca de una
jugada.
\begin{itemize}
\item \textbf{Atributos}: el nodo inicial a partir del cual
  comenzaremos la búsqueda, el mejor nodo encontrado hasta el momento
  y una variable de control que nos indica si podemos seguir buscando
  o no.
\item \textbf{Métodos}.
  \begin{itemize}
  \item \emph{Sync()} reinicia la variable compartida \emph{timeout} a
    ``falso'' y duerme durante X segundos, donde X es el tiempo por
    turno con el que se configura la partida menos un breve intervalo
    de tiempo de seguridad.
  \item \emph{decision-minimax()} inicia la búsqueda en profundidad
    iterativa desde el nodo inicial expandiéndolo como MAX.
  \item \emph{max-valor()} y \emph{min-valor()} son los métodos
    utilizados en el algoritmo minimax para la búsqueda de los mejores
    y peores sucesores. Más adelante se explicará con detalle.
  \item \emph{expandir()} expande un nodo generando únicamente los
    sucesores válidos que han generado algún cambio en el estado.
  \item \emph{test-terminal()} devuelve ``verdadero'' si el nodo
    contiene un estado objetivo o si se ha llegado a la profundidad de corte.
  \item \emph{evaluacion()} obtiene un valor para el nodo dado. Esta
    función será explicada con detalle más adelante.
  \end{itemize}
\end{itemize}

\imagen{class_diagram_minimax}{9cm}{Clase Minimax}{classdiagramminimax}

\label {sec:soluciones_teoricas}
\chapter {Soluciones Teóricas}
En un algoritmo Mini-Max se consideran dos jugadores a los que
llamaremos Max y Min ya que, una vez que evaluamos un nodo, se supone
que los valores altos son buenos para Max y los malos para Min. De
esto se deduce que, dado que queremos que nuestro equipo consiga las
mejores puntuaciones, nosotros seremos el equipo Max y nuestro rival
será el equipo Min.

Considerando un árbol de juegos, la estrategia óptima puede
determinarse examinando el \textbf{valor minimax} de cada nodo (si es un estado
terminal es solamente su utilidad). Un algoritmo minimax calcula la
decisión minimax del estado actual. La recursión avanza hacia las
hojas del árbol, y entonces los valores minimax retroceden por el
árbol cuando la recursión se va deshaciendo. Como se puede deducir, un
algoritmo minimax realiza una \textbf{búsqueda primero en profundidad}
completa del árbol de juegos, es decir, si la profundidad máxima de
árbol es \emph{m}, y hay \emph{b} movimientos legales en cada punto,
entonces la complejidad en tiempo del algoritmo minimax es $O(b^m)$ y
la complejidad en espacio es $O(bm)$.

Como hemos visto, el problema de la búsqueda minimax es que el número
de estados que tiene que examinar es exponencial en el número de
movimientos. Lamentablemente no podemos eliminar el exponente, pero
podemos dividirlo, con eficacia, en la mitad. La jugada es que es
posile calcular la decisión minimax correcta sin mirar todos los nodos
en el árbol de juegos aplicando la ténica de \textbf{poda alfa-beta}. Esta poda
puede aplicarse en árboles de cualquier profundidad y, a menudo, es
posible podar subárboles enteros. Además, es necesario tener en cuenta
que los estados repetidos en el árbol de búsqueda pueden causar un
aumento exponencial del coste de búsqueda. Esto se debe a
permutaciones diferentes de la secuencia de movimientos que terminan
en la misma posición. Para resolver este problema se utiliza una \textbf{tabla
de transposición}, que tradicionalmente es una \emph{tabla hash} idéntica a la
\emph{lista cerrada} que utilizamos en el algoritmo A* de la primera
práctica. La forma de trabajar con dicha tabla es guardando la
evaluación de cada posición la primera vez que se encuentre, de modo
que no tenemos que volver a calcularla las siguientes veces.

Por último, hay que tener en cuenta que la \textbf{función de utilidad} (que
comprueba si un estado es objetivo y le asigna un valor de utilidad)
hay que modificarla ligeramente si no podemos explorar completamente
el árbol de búsqueda (y no podremos ya que los turnos están limitados
en tiempo), ya que no podemos explorar las ramas hasta las hojas y
debemos cortar a una determinada profundidad. Esta nueva función
recibe el nombre de \textbf{función de evaluación}, que devuelve una estimación
de la utilidad esperada de una posición dada. En este momento, podemos
plantearnos la posibilidad de implementar nuestro algoritmo mediante
\textbf{profundidad iterativa} para que cuando se agote el tiempo, el programa
nos devuelva el movimiento seleccionado por la búsqueda completa más
profunda.

\section{Conceptos básicos}
Para la realización de esta práctica es necesario tener en cuenta los
siguientes conceptos:
\begin{itemize}
\item \textbf{Estado}. En la primera práctica un estado estaba
  compuesto por una lista de jugadores (ya que únicamente jugaba un
  equipo) y una lista de casillas del tablero que se habían ido
  modificando a lo largo de la partida. En esta segunda práctica, en
  la que juegan juegan dos equipos (el nuestro y el rival), es
  necesario hacer un pequeño ajuste para que el estado esté compuesto
  por una lista de equipos (cada equipo tendrá una lista con sus
  jugadores) y una lista de casillas modificadas. Pero, como hemos
  visto en la sección anterior, la complejidad en espacio de un
  algoritmo minimax es $O(bm)$. Dicho esto, puede ser interesante
  plantearse la posibilidad de almacenar el estado del tablero
  completo para no tener que recalcularlo cada vez que lo necesitemos
  aplicando los cambios indicados por la lista de casillas modificadas.
\item \textbf{Función sucesor}. Esta función es similar a la que
  utilizamos en la primera práctica, solo que añadiremos un séptimo
  movimiento para poder cavar. De modo que serán dos bucles anidados
  en los que el primero recorrerá la lista de jugadores y el segundo
  hará cada uno de los siete posibles movimientos, generando así todos
  los posibles sucesores de un nodo.
\item \textbf{Test terminal}. Esta función determina si un nodo contiene un
estado objetivo. Un estado será objetivo cuando se hayan capturado
todas las banderas del equipo contrario o no nos quede ningún jugador
con vida.
\item \textbf{Función de utilidad}. Nuestro juego es un \textbf{juego de
  suma no cero}. Esta función evaluará un nodo terminal y le asignará
un valor de utilidad en función del número de banderas capturadas y
la energía total de cada equipo, tratando así de maximizar las
banderas que captura nuestro equipo y su energía; y de minimizar las
banderas que captura el equipo rival y su energía. En esta función no
es necesario tener en cuenta las distancais de los jugadores a las
banderas restantes ya que estamos hablando de \textbf{nodos terminales}.
\item \textbf{Función de evaluación}. Convierte los nodos no terminales en
hojas terminales. Es necesario sustituir test terminal por un
\textbf{test-límite} que decide cuando aplicar la función de
evaluación (que sustituye a la función de utilidad), que nos devolverá
una estimación de la utilidad esperada de una posición dada.
\end{itemize}

\section {Algoritmo Minimax}
\label{minimax}
A continuación se muestra el pseudocódigo del algoritmo Mini-Max que
se implementará. Este pseudocódigo está diseñado para una
\emph{búsqueda primero en profundidad} por lo que habrá que hacer una
pequeña modificación en la función \emph{decisión-minimax()} para que
la búsqueda sea de \emph{profundidad iterativa}. Esta modificación
consiste en pasar a \emph{valor-max()} y \emph{valor-min()} un
argumento \emph{límite} que se irá incrementando unidad a unidad
mediante una estructura iterativa de tipo \emph{for}. Las funciones
\emph{valor-max()} y \emph{valor-min()} pasarán a su vez este
parámetro al \emph{test-terminal()} que comprobará cuándo se llega a
la profundidad de corte.

La documentación utilizada a la hora de diseñar el algoritmo Mini-Max
e incorporarle la poda Alfa-Beta se puede encontrar en
\cite{russell03}

Nótese que a este nivel, la función \emph{utilidad()} es una mera
aproximación y será revisada con más detenimiento en la sección
\ref{heuristicas}.

\texttt{\lstinputlisting[inputencoding=utf8]{documentos/minimax.txt}}

\section {Poda Alfa-Beta}
A continuación se muestra el pseudocódigo del algoritmo minimax de la
sección \ref{minimax} con la poda alfa-beta incorporada.

Su funcionamiento es sencillo. Consiste en dos variables locales
\emph{alfa} y \emph{beta} que indican el valor de la mejor alternativa
para MAX y el valor de la mejor alternativa para MIN a lo largo del
camino respectivamente. Inicialmente, \emph{alfa} es tiene un valor
muy malo para que éste pueda ser mejorado (-INFINITO), y \emph{beta}
tiene un valor muy bueno para que éste pueda ser empeorado (INFINITO),
ya que a MAX le interesa que el contrario obtenga el menor
beneficio. Estas variables se van pasando de una llamada a otra y sus
valores van siendo actualizados. De este modo podemos detectar cuando
hay subárboles que no son prometedores y podemos ``podarlos'' sin
compromiso alguno.

\texttt{\lstinputlisting[inputencoding=utf8]{documentos/alfabeta.txt}}

\section {Heuristicas Diseñadas}
\label{heuristicas}
En realidad sólo se ha diseñado una heurística, pero ésta se ha
obtenido a lo largo de cuatro iteraciones bien definidas. En cada
iteración se trataba de lograr un objetivo, es decir, un
comportamiento por parte de los jugadores

\subsection {Algoritmo WayTracking}
\label{alg:waytracking}
Probablemente el aspecto más crucial a la hora de desarrollar la
heurística sea el cálculo de distancias reales desde los jugadores a
las banderas. Este problema ya se planteó en la primera práctica
cuando se desarrolló el algoritmo de \emph{búsqueda informada A*} y se
resolvió mediante distancias euclídieas o distancias en línea recta,
una forma muy sencilla y válida en caso de tratarse de un entorno
\textbf{parcialmente observable}, pero como en este caso el entorno es
\textbf{totalmente observable}, se trata de un cálculo poco eficiente
ya que no tiene en cuenta las restricciones que pueda imponer éste (ver figura
\ref{'waytracking'}). Por esta razón, se propone un algoritmo
desarrollado por el autor de este documento y que ha sido denominado
\textbf{WayTracking}. WayTracking es un algoritmo de tipo
\emph{backtracking} que calcula las distancias reales desde una
casilla a todas las demás casillas teniendo en cuenta las murallas del
tablero. Se trata de un algoritmo muy eficiente y se ha conseguido
calcular las distancias en mapas de hasta $100x100$ casillas en menos
de $0.5$ segundos.

\imagen{waytracking}{6cm}{Cálculo de distancias reales y en línea
  recta}{waytracking}

WayTracking consta de dos fases:
\begin{enumerate}
\item La primera es una inicialización en la que se etiqueta la
  casilla destino donde se encuentra la bandera con un cero, indicando
  distancia cero. A continuación, se etiquetan todas las casillas del
  tablero a una distancia máxima de la casilla destino, por ejemplo
  INFINITO. El siguiente paso es etiquetar todas las casillas que son
  de tipo \emph{muralla} a distancia $-1$. De este modo se indica que
  esas casillas son inalcanzables. Dada esta fase de inicialización,
  WayTracking sólo puede utilizarse en entornos totalmente observables.
\item La segunda fase son una serie de llamadas recursivas al
  algoritmo de WayTracking que se encarga de calcular las distancias
  mínimas desde la casilla destino a cada una de las casillas
  etiquetadas como INFINITO del tablero.
\end{enumerate}

El algoritmo se planteó siguiendo los siguientes razonamientos:
\begin{itemize}
\item En la \textbf{etapa} $k$ etiquetamos las casillas adyacentes a
  las casillas a distancia $k$ de la bandera supuesto que hemos
  etiquetado las $k-1$ casillas anteriores, de modo que las casillas
  etiquetadas con un valor inferior a $k$ ya están etiquetadas con su
  distancia mínima.
\item La \textbf{generación de descendientes} consiste en recorrer con un bucle
  \emph{for} todas las casillas adyacentes a aquellas que tienen distancia
  $k$.
\item El \textbf{test de solución} es cuando hemos etiquetado todas las casillas
($filas*columnas$)
\item El \textbf{test de fracaso} es intentar etiquetar con una distancia mayor
  una casilla etiquetada o etiquetar una muralla. Como las murallas se
  inicializaron con distancia $-1$, no podemos etiquetarla en ningún
  momento porque $k$ comienza en cero.
\item Buscamos minimizar la distancia, así que las \textbf{inicializaciones} se
  harán a INFINITO. No se trata de un problema de optimización de
  soluciones, por lo que no habrá que comparar distintas
  soluciones.
\item La \textbf{solución} será una lista que indica la distancia real de cada
  casilla a una casilla dada.
\end{itemize}

Dicho esto, podemos observar la importancia de la información que nos
genera WayTracking obsevando la figura \ref{'waytracking'}, donde
podemos ver que la distancia en línea recta es de $2$ casillas
mientras que WayTracking indica que son $4$ casillas. Gracias a esto,
podemos saber realmente si nuestros jugadores están más cerca de las
banderas que los jugadores contrarios.

Como WayTracking sólo calcula las distancias desde una casilla hasta
todas las demás casillas del tablero, deberemos hacer una ejecución
del algoritmo por cada bandera. Así, podremos hacer un diccionario de
distancias donde podremos encontrar las distancias reales desde
cualquier casilla hasta cualquier bandera. Además, como el tablero es
completamente obsevable antes de que comienze la partida (justo al
conectar al servidor) y las banderas siempre están en posiciones
fijas, podemos lanzar las ejecuciones de WayTracking antes de consumir
tiempo de nuestro turno y realizar los cálculos una única vez,
reutilizándolos a lo largo de toda la partida.

La implementación de WayTracking puede verse en el código fuente de la
clase Tablero (sección \ref{clasetablero}).

\subsection {Primera aproximación}
\label{sec:distancias}
En una primera iteración del diseño de la heurística, se ha tratado
que los jugadores vayan a coger las banderas que se encuentran a menor
distancia sin tener en cuenta la posición de los jugadores contrarios.

\imagen{func_exp}{10cm}{$b*e^{(-1/b)*x}$, siendo $b=5$}{funcexp}

\imagen{func_lineal}{10cm}{$b-x$, siendo $b=5$}{funclineal}

Es aquí cuando se plantea un problema a la hora de valorar las
bondades de las distancias y las banderas porque, cuantas más banderas
capturemos mejor, pero cuanto mayores sean las distancias mínimas peor
es nuestra posición de juego. Con esto se intenta ver que hay que dar
mayor valor a las distancias bajas y para ello necesitamos alguna
función que nos invierta este concepto. Se proponen dos funciones como
caso de estudio: $b*e^{(-1/b)*x}$ (ver figura \ref{'funcexp'}) y $b-x$
(ver figura \ref{'funclineal'}).

Vamos a tratar primero el tema de las banderas capturadas ya que es
más sencillo. En esta primera interación se obtenían las banderas
capturadas por el equipo MAX y las banderas capturadas por el equipo
MIN para hacer la diferencia que se multiplicaba por una constante.

Para tratar el tema de las distancias es necesario entender las
gráficas mostradas en \ref{'funcexp'} y \ref{'funclineal'}: el eje de
abcisas representa la distancia que estamos midiendo, y el eje de
ordenadas representa el valor que vamos a darle a esa distancia. De
este modo podemos obtener el cuadro \ref{tab:valoraciones}.

Las ventajas de \ref{'funclineal'} sobre \ref{'funcexp'} es que con
un mismo parámetro, no tiende hacia un valor 0 para distancias muy
grandes y hace que las valoraciones sean más significativas porque da
valores enteros y no valores con decimales que hacen que éstos sean
muy próximos entre sí.

\begin{table}[h!]
  \centering
  \begin{tabular}[h!]{|c|c|c|}
    \hline
    \textbf{Distancia} & \textbf{Función \ref{'funcexp'}} & \textbf{Función
    \ref{'funclineal'}} \\ \hline
    5 & 1.83 & 0.00 \\ \hline
    4 & 2.24 & 1.00 \\ \hline
    3 & 2.74 & 2.00 \\ \hline
    2 & 3.35 & 3.00 \\ \hline
    1 & 4.09 & 4.00 \\ \hline
    0 & 5.00 & 5.00 \\ \hline
  \end{tabular}
  \caption{Valoraciones de las funciones \ref{'funcexp'} y
    \ref{'funclineal'} para $b=5$}
  \label{tab:valoraciones}
\end{table}

Una vez que tenemos las valoraciones de las distancias mínimas de los
jugadores MAX y MIN, las multiplicamos por una constante, calculamos
la diferencia y lo sumamos a la evaluación que habíamos obtenido de
las banderas. 

Hasta ahora hemos evaluado los dos equipos del mismo modo y la
evaluación resultante es la diferencia de ambas evaluaciones, pero
existe el problema de las constantes utilizadas al evaluar las
banderas y las distancias que deberá ser resuelto en la segunda
iteración (ver sección \ref{sec:bondades}).

Por último, otro factor a tener en cuenta es que no midamos la
distancia mínima desde un jugador que no está muerto pero que no se
puede mover (p.e. un jugador rodeado de agua con 2 unidades de
energía). Esto se resuelve mediante la función
\emph{jugadorBloqueado()} de la \emph{Clase Estado}, que nos indica si
el jugador se puede mover o no: si puede moverse lo tendremos en
cuenta para el cálculo de distancias mínimas; en caso contrario, se
considerará que está muerto.


\subsection {Segunda aproximación}
\label{sec:bondades}
En esta iteración definiremos cuál será el parámetro $b$ que tanto
condiciona nuestra función \ref{'funclineal'}. Después de una serie de
pruebas, se ha aproximado que este parámetro tendrá un valor igual a
$\frac{columnas+filas}{2}*\frac{3}{4}$, dado que es una aproximación a
la máxima distancia que puede haber en cualquier tipo de tablero.

Ahora, dado que la bondad de las distancias es variable, tenemos que
asegurarnos que las distancias no obtengan mejores valoraciones que la
captura de banderas. Esto se consigue sustituyendo la constante que
indica la bondad de las banderas por otro valor que varie en función
del tablero. Se propone utilizar un valor igual a $filas*columnas$.


\subsection {Tercera aproximación}
\label{sec:blacklist}
Hasta ahora nuestros jugadores capturan correctamente las banderas
cualesquiera que sea la dimensión del tablero y siempre y cuando
nuestro contricante no coja las banderas antes que nosotros. Esto
plantea el problema de que nuestros jugadores van a por la bandera con
distancia más corta si pensar que el contrario va también a por esa misma
bandera, dado que también es la que tiene a mínima distancia.

Aquí es donde entra el concepto de \textbf{blacklist} o \textbf{lista
  negra}. Consiste en que una vez que se han calculado las distancias
mínimas de ambos equipos, se comprueban si ambas distancias se
refieren a la misma bandera. En caso afirmativo, si la distancia del
equipo contrario es menor que la de nuestro equipo, esa bandera se
añadirá a la lista negra y se volverá a calcular la distancia mínima
para nuestro equipo sin tenerla en cuenta. Este proceso se
hará hasta que se encuentre una bandera a la que realmente podamos
optar a capturar o, si se da el caso en que el jugador contrario está
mejor situado que nosotros con respecto a todas las banderas y todas
ellas se han añadido a la lista negra, se escogerá una bandera
aleatoriamente y será removida de la lista negra para dirigirnos hacia
ella.

Si obsevamos la figura \ref{'blacklist'}, se aprecia que la distancia
mínima del \emph{Jugador 1} es 1 y se refiere a la \emph{Bandera 1}, y
que la distancia mínima del \emph{Jugador 2} es 2 y también se refiere
a la \emph{Bandera 1}. Como el \emph{Jugador 2} no puede optar a coger
la \emph{Bandera 1}, la añadirá a su \emph{blacklist} y volverá a
calcular su distancia mínima, siendo en este caso 3 refiriéndose a la
\emph{Bandera 2}. Como esta bandera no está amenazada, irá a
capturarla dejando que el \emph{Jugador 1} capture la \emph{Bandera
  1}.

\imagen{blacklist}{14cm}{Ejemplo de \emph{lista negra}}{blacklist}

\subsection {Cuarta aproximacion}
\label{sec:energia}
Se llevó a cabo una cuarta iteración en la que se trataba de ahorrar
la energía consumida. Se consiguió este comentido pero si no se
llegaba en la búsqueda a la profundidad necesaria, podía provocar que
perdiésemos alguna bandera o incluso la partida, por lo que fue
deshechada.

Si observamos la figura \ref{'savingenergy'} podemos ver un claro
ejemplo en el que ahorrar energía nos podría suponer la pérdida de una
bandera (suponiendo que mueva primero el \emph{Jugador 1}).

\imagen{savingenergy}{14cm}{Ejemplo de ahorro de
  energ{í}a}{savingenergy}

Para evitar este tipo de situaciones, ya que los zumos creaban aún más
conflicto, esta característica ha sido suprimida de la heurística.

\subsection {Heurística final}
Después de estas cuatro iteraciones, la heurística que hemos obtenido
tiene las siguientes características:
\begin{itemize}
\item Siempre trabaja con distancias reales independientemente de las
  restricciones que imponga el trablero gracias al algoritmo
  \textbf{WayTracking}. Como ya se ha explicado, trabajar con
  distancias reales en lugar de distancias estimadas en línea recta es
  una solución mucho más eficiente (ver sección \ref{alg:waytracking}).
\item No sólo se ignoran jugadores muertos en el cálculo de distancias
  mínimas, si no que también se ignoran jugadores que no pueden
  moverse por la razón que sea (ver sección \ref{sec:distancias}).
\item Evalúa el número de banderas y las distancias mínimas a las que
  se encuentran los jugadores de cada equipo utilizando bondades que
  varían en función del tamaño del tablero (ver sección \ref{sec:bondades}).
\item Tiene en cuenta que las banderas a las que se refieren las
  distancias mínimas de cada equipo no sean las mismas, para no tratar
  de coger una bandera que sabemos que capturará el otro equipo antes
  que nosotros. Esto es gracias al uso de la \textbf{lista negra} (ver
  sección \ref{sec:blacklist}).
\end{itemize}

\label {sec:pruebas}
\chapter {Pruebas}
\label {pruebas}
\section {Detección de estados repetidos}
Como ya se comentó en la sección \ref{sec:estadosrepetidos} se ha
tratado de realizar un control sobre los estados repetidos. Se trata
de una característica importante porque en este tipo de problemas el
número de estados a expandir y examinar crece exponencialmente y, si
evitamos situaciones iguales podemos reducir drásticamente este
espacio de búsqueda.

Mantener una estructura de datos que almacene los nodos que ya hemos
expandido o examinado no sólo es costoso en términos de memoria si no
que también lo es en términos de tiempo. No obstante, dado que los
turnos están limitados por tiempo, la memoria no será un problema a no
ser que los turnos sean lo suficientemente largos como para generar
una grandísima cantidad de nodos. Por ello, nos hemos centrado en la
optimización de tiempo por medio de dos técnicas:
\begin{itemize}
\item Una lista.
\item Un diccionario simulando una tabla hash.
\end{itemize}

Inicialmente y sabiendo que era la peor solución, se trató de resolver
el problema mediante una lista de estados. La peor parte de esta
solución viene dada cuando la lista contiene miles de elementos y
queremos comprobar si un estado ya está contenido en la lista. La
comparación de estos elementos uno a uno es muy tediosa ya que
interviene un gran número de atributos (de casillas, jugadores,
equipos...) y hay que realizar esta operación miles de veces. Tras
realizar las pruebas pertinentes, se comprobó que no obteníamos
mejores niveles de profundidad y que en ocasiones éstos empeoraban.

Por último, se trató de implementar una tabla hash por medio de un
diccionario. Esta solución parece más práctica puesto que dado un
identificador, podemos acceder a ese elemento directamente. El
problema está en qué identificador utilizar. Se ha tratado de
identificar cada estado mediante una máscara formada por las
características del estado, es decir, una cadena de dígitos que
representan las casillas, los jugadores, los equipos, etc. pero, a
parte de ser un método que puede llevar a confusión en la
identificación de estados, no ha dado los resultados esperados en
cuanto a los niveles de profundidad alcanzados en la búsqueda. Éstos
son similares a los obtenidos sin controlar los estados repetidos dada
la poca duración de los turnos.

En conclusión, esta característica ha sido contemplada, estudiada y
rechazada.

\section {Resultados prácticos}
A continuación se comentarán los resultados prácticos obtenidos
durante la ejecución de las pruebas del programa en las distintas
fases de desarrollo de la heurística. Los escenarios de prueba han
sido los mapas \emph{Mapas de pruebas mejor} y \emph{Mapa},
disponibles en la página web de la asignatura.

\subsection{Primera iteración}
Esta primera iteración corresponde con la heurística desarrollada en
la primera aproximación (ver sección \ref{sec:distancias}).

Tras una serie de tests y ajustes, el comportamiento del bot era el
esperado en mapas de dimensiones reducidas: los jugadores se
aproximaban a las banderas hasta capturarlas. En mapas de dimensiones
algo mayores, los jugadores hacían movimientos sin control debido a
que los parámetros que condicionan las funciones \ref{'funcexp'} y
\ref{'funclineal'} no habían sido ajustados. La consecuencia de esto
es que un estado con una distancia mínima de 6 obtenía la misma
valoración que otro estado con una distancia mínima de 10, cuando éste
segundo estado es claramente peor.

\subsection{Segunda iteración}
En esta segunda iteración, que corresponde con la segunda aproximación
de la heurística (ver sección \ref{sec:bondades}), se hicieron un
cálculos sencillos para determinar los valores que pueden tomar los
parámetros que condicionan las funciones \ref{'funcexp'} y
\ref{'funclineal'} para que éstas valoren bien las distancias
cualesquiera que sean las dimensiones del tablero.

Los resultados obtenidos fueron los esperados: se ha resuelto en
problema que se planteaba en la primera iteración cuando las
dimensiones del mapa eran lo suficientemente grandes. En este momento,
los jugadores ya se dirigen a las banderas y las capturan,
independientemente de las dimensiones del tablero.

\subsection{Tercera iteración}
Esta tercera iteración, que corresponde con la tercera aproximación de
la heurística (ver sección \ref{sec:blacklist}), ha sido la más
compleja de testear y depurar para obtener los resultados
esperados. No obstante, se ha logrado dotar a los jugadores con algo
más de inteligencia para que éstos solo traten de capturar banderas
que, en la medida de lo posible, tienen probabilidades de capturar,
es decir, no compiten por banderas que es seguro que no van a poder
capturar porque el contrincante las capturará antes.

\subsection{Cuarta iteración}
Esta iteración se correponde con la cuarta aproximación de la
heurística (ver sección \ref{sec:energia}). Los resultados obtenidos
eran los esperados si el nivel de profundidad al que se llegaba en la
búsqueda era el adecuado ya que los jugadores se movían por los
caminos menos costosos, pero si no se alcanzaba la profundidad
adecuada, ahorrar energía podía suponer aumentar la longitud del
camino y perder una bandera y, en el peor de los casos, perder la
partida. Para evitar este tipo de situaciones, esta característica ha
sido suprimda de la heurística.

\label {sec:manual_usuario}
\chapter {Manual de Usuario}

En esta sección se darán unas breves nociones de cuál es el contenido
de la práctica, qué es necesario instalar en el sistema para poder
ejecutarla y cómo ejecutarla.

\section{Organización de directorios}
La práctica se distribuye en un fichero con extensión \emph{.tar.gz}
que contiene cinco directorios:
\begin{itemize}
\item \textbf{config}. Contiene un fichero \emph{config.client}
  empleado por el middleware de comunicaciones ICE. Se recomienda no
  modificar este fichero.
\item \textbf{doc}. Contiene la documentación de la práctica así como
  las imágenes utilizadas.
\item \textbf{exec}. Contiene dos scripts bash \emph{run} y \emph{kill}.
  \begin{itemize}
  \item \textbf{run}. Admite un parámetro que es un número entero en
    el cual se indica el número de jugadores que van a competir (1 si
    es una partida individual y 2 si se trata de una competición). Si
    no se suministra este parámetro, se mostará un mensaje de
    ayuda. Ejemplo de uso: \emph{./run 1}
  \item \textbf{kill}. Su ejecución no admite ningún parámetro
    (\emph{./kill}). Consiste en una tubería compuesta mediante las
    utilidades \emph{ps}, \emph{grep} y \emph{awk} que obtiene los
    \emph{PID} de los clientes en ejecución y los mata. Es un script
    muy cómodo en el proceso de depuración de clientes.
  \end{itemize}
\item \textbf{slice}. Contiene un fichero \emph{Practica.ice} que
  indica los tipos de datos e interfaces públicos del servidor ICE de
  la práctica.
\item \textbf{src}. Contiene todos los ficheros que forman parte del
  código fuente de la práctica. Se puede observar un fichero para cada
  clase y dos ficheros (global-vars.py y config.py) que definen las
  variables globales y las constantes.
\end{itemize}

\section {Instalación}
Antes de ejecutar la práctica hay que instalar tres componentes en el
sistema:
\begin{itemize}
\item \textbf{python} \cite{python}. Es el intérprete de python.
\item \textbf{psyco} \cite{psyco}. Es un compilador en tiempo de ejecución
  especializado para python. Psyco puede acelerar notablemente
  aplicaciones que hacen un uso intensivo de la CPU. El rendimiento
  actual depende de forma importante de la aplicación y pueden
  aumentarse hasta 40 veces, aunque la mejora de rendimiento media es
  aproximadamente de 4x. Puede descargarlo de 
\item \textbf{ZeroC Ice} \cite{zeroc}. Runtime del middleware de comunicaciones
  empleado por el cliente y el servidor de la práctica.
\end{itemize}
Si se goza de una distribución GNU/Linux basada en debian
simplemente hay que instalar los metapaquetes \textbf{python},
\textbf{python-psyco} y \textbf{zeroc-ice33}

\section {Ejecución}
Para poder ver como juega el \emph{bot} es necesario configurar la
partida en la web de la asignatura \cite{webia} siguiendo los
siguientes pasos:
\begin{enumerate}
\item Abrir la página de la asignatura \cite{webia}.
\item Introducir en la parte superior derecha nuestro login (DNI) y
  password para realizar la autenticación e iniciar una sesión.
\item Clickar en \textbf{Partida Activa} para acceder al panel de
  configuración de partidas (ver figura \ref{'paneladmon'}) si no hay ninguna partida creada, o
  visualizar la partida en curso.
\imagen{webia_panel_admon}{14cm}{Panel de Administración}{paneladmon}
\item Si queremos crear una partida: elegimos el mapa, el tiempo por
  turno y el modo de juego (individual o competición); si por el
  contrario queremos unirnos a una partida ya creada (modo
  competición), elegimos a qué partida deseamos unirnos.
\end{enumerate}

Una vez que clickemos en \textbf{Aceptar} y ya tenemos configurada nuestra
partida. Ahora tenemos que ejecutar el cliente para conectarnos al
servidor. Para ello seguimos la siguiente secuencia de pasos:
\begin{enumerate}
\item Descomprimimos la práctica con el comando \emph{tar xvfz
    fichero-practica.tar.gz}.
\item Nos situamos en el directorio \emph{exec}.
\item Si vamos a jugar una partida invididual ejecutamos el script
  \emph{run} con el argumento \emph{1}. Abrimos otro terminal y
  repetimos este mismo paso; Si por el contrario vamos a jugar una
  partida en modo competición, ejecutamos el script \emph{run} con el
  argumento \emph{2}.
\item El programa conectará con el servidor y cuando ambos jugadores
  estén conectados lanzarán sus algoritmos minimax y comenzarán la partida.
\end{enumerate}


\label {sec:conclusiones}
\chapter {Conclusiones}

Después de la realización de esta práctica se ha comprendido cómo
funciona un algoritmo Mini-Max y las ventajas que aporta la
incorporación de la poda Alfa-Beta. Además, se han reutilizado los
conceptos aprendidos durante la realización de la primera práctica
para la elección del método de búsqueda: \emph{primero en profundidad} o
\emph{profundidad iterativa}.

Como se ha podido comprobar a lo largo de las pruebas que se han hecho
con el cliente, los papeles más importantes a la hora de decidir la
mejor acción que se puede realizar en cada momento los juegan la
heurística y la profundidad a la que llegamos explorando. Sin duda, el
comportamiento del bot viene determinado por la heurística (p.e. ignorar
banderas que sabemos que no vamos a coger) pero la profundidad a la
que llegamos en la búsqueda es tan importante como la heurística ya
que podemos obtener comportamientos inesperados debido a situaciones
que no son del todo estables (p.e. perder una bandera por tratar de
ahorrar energía se podría haber evitado si el nivel de profundidad al
que llegamos explorando incluye la captura de esa bandera). Este
último aspecto hace todavía más importante la incorporación de la poda
Alfa-Beta al algoritmo Mini-Max, ya que nos permite la poda de
sub-árboles completos y, por consiguiente, invertir el tiempo que
invertiríamos en explorar esos sub-árboles en explorar otras ramas a
mayores niveles de profundidad.

Por último, y en concepto de resumen, en este documento se han tratado
los siguientes temas:
\begin{itemize}
\item En qué consiste el problema propuesto, las decisiones de diseño
  que se han tomado y las estructuras de datos que se han empleado
  para su resolución.
\item Las soluciones teóricas propuestas, cómo son llevadas a la
  práctica, y cómo realizar una buena heurística en varias iteraciones
  bien definidas, dividiendo el problema y resolviéndolo por partes.
\item Los comportamientos, tanto esperados como inesperados, que se
  han obtenido por parte del bot en la ejecución de las pruebas.
\item Y un manual de usuario para saber qué software es necesario
  instalar en el sistema y cómo se debe ejecutar la práctica para
  verla en funcionamiento.
\end{itemize}

Dejando de lado aspectos de los que ya se hablan en otras secciones de
este documento y refiriéndome a experiencias personales, tengo que
decir que esta es la primera vez que escribo una documentación en
\LaTeX{} y, aunque he tardado más tiempo en escribirla que con otros
procesadores de textos como \emph{{O}pen{O}ffice.org {W}riter}, considero que
la presentación y calidad del documento merecen la pena.

Por último, quiero agradecer a los profesores y desarrolladores del
servidor la inclusión de distintos clientes en diferentes lenguajes de
programación ya que nos sirven de ejemplo para futuros usos del
interfaz de comunicaciones {I}ce.

\label {sec:codigo_fuente}
\chapter {Código Fuente}

A continuación se mostrará el código fuente de cada uno de los
ficheros que forman parte de la práctica.
\section {Variables Globales}
\texttt{\lstinputlisting[inputencoding=utf8]{../../src/global_vars.py}}

\section {Constantes}
\texttt{\lstinputlisting[inputencoding=utf8]{../../src/config.py}}

\section {Clase Casilla}
\texttt{\lstinputlisting[inputencoding=utf8]{../../src/casilla.py}}

\section {Clase Jugador}
\texttt{\lstinputlisting[inputencoding=utf8]{../../src/jugador.py}}

\section {Clase Equipo}
\texttt{\lstinputlisting[inputencoding=utf8]{../../src/equipo.py}}

\section {Clase Tablero}
\label{clasetablero}
\texttt{\lstinputlisting[inputencoding=utf8]{../../src/tablero.py}}

\section {Clase Estado}
\texttt{\lstinputlisting[inputencoding=utf8]{../../src/estado.py}}

\section {Clase Nodo}
\texttt{\lstinputlisting[inputencoding=utf8]{../../src/nodo.py}}

\section {Clase Minimax}
\texttt{\lstinputlisting[inputencoding=utf8]{../../src/minimax.py}}

\section {Clase Cliente}
\texttt{\lstinputlisting[inputencoding=utf8]{../../src/client.py}}


\renewcommand{\baselinestretch}{1}  %---- Modo de interlineado -------------
\large\normalsize

\bibliographystyle{plain} 
\bibliography{Documentacion}

\end{document}


% ---------------- Ejemplo de c�mo insertar un PDF o una imagen ----------------
% ------------------------------------------------------------------------------
% \imagen{personajeEsqueleto}{14cm}{Imagen JPG de ejemplo}
%        {FIGpersonaje}


% ------------------------ Ejemplo de cita de un libro -------------------------
% ------------------------------------------------------------------------------
% \cite{id_del_bibitem}


% --------------------- Ejemplo de referencia a una imagen ---------------------
% ------------------------------------------------------------------------------
% (ver imagen \ref{'label_de_la_imagen'}), n�tese el uso de comillas simples


% ------------------------- Insertar el logo de LaTeX --------------------------
% ------------------------------------------------------------------------------
% \LaTeX{}


% -- Ejemplos de ecuaciones (nótese el $ al final y al inicio de la ecuación)---
% ------------------------------------------------------------------------------
% $x^2+y^2=1$

% \begin{eqnarray} \label{EQContinueLingLabel}
% \sum_{SA^{i}_{j} \in SA_{j}}\mu_{SA^{i}_{j}}(x)=1
% \end{eqnarray}

% $X_{j}$

% Para citar una ecuación se hará mediante \ref{EQContinueLingLabel},
% sin comillas

% --------------------------- Ejemplo para insertar código fuente --------------
% ------------------------------------------------------------------------------
% \texttt{
% \lstinputlisting{../src/casilla.py}}