\chapter {Conclusiones}

Después de la realización de esta práctica se ha comprendido cómo
funciona un algoritmo Mini-Max y las ventajas que aporta la
incorporación de la poda Alfa-Beta. Además, se han reutilizado los
conceptos aprendidos durante la realización de la primera práctica
para la elección del método de búsqueda: \emph{primero en profundidad} o
\emph{profundidad iterativa}.

Como se ha podido comprobar a lo largo de las pruebas que se han hecho
con el cliente, los papeles más importantes a la hora de decidir la
mejor acción que se puede realizar en cada momento los juegan la
heurística y la profundidad a la que llegamos explorando. Sin duda, el
comportamiento del bot viene determinado por la heurística (p.e. ignorar
banderas que sabemos que no vamos a coger) pero la profundidad a la
que llegamos en la búsqueda es tan importante como la heurística ya
que podemos obtener comportamientos inesperados debido a situaciones
que no son del todo estables (p.e. perder una bandera por tratar de
ahorrar energía se podría haber evitado si el nivel de profundidad al
que llegamos explorando incluye la captura de esa bandera). Este
último aspecto hace todavía más importante la incorporación de la poda
Alfa-Beta al algoritmo Mini-Max, ya que nos permite la poda de
sub-árboles completos y, por consiguiente, invertir el tiempo que
invertiríamos en explorar esos sub-árboles en explorar otras ramas a
mayores niveles de profundidad.

Por último, y en concepto de resumen, en este documento se han tratado
los siguientes temas:
\begin{itemize}
\item En qué consiste el problema propuesto, las decisiones de diseño
  que se han tomado y las estructuras de datos que se han empleado
  para su resolución.
\item Las soluciones teóricas propuestas, cómo son llevadas a la
  práctica, y cómo realizar una buena heurística en varias iteraciones
  bien definidas, dividiendo el problema y resolviéndolo por partes.
\item Los comportamientos, tanto esperados como inesperados, que se
  han obtenido por parte del bot en la ejecución de las pruebas.
\item Y un manual de usuario para saber qué software es necesario
  instalar en el sistema y cómo se debe ejecutar la práctica para
  verla en funcionamiento.
\end{itemize}

Dejando de lado aspectos de los que ya se hablan en otras secciones de
este documento y refiriéndome a experiencias personales, tengo que
decir que esta es la primera vez que escribo una documentación en
\LaTeX{} y, aunque he tardado más tiempo en escribirla que con otros
procesadores de textos como \emph{{O}pen{O}ffice.org {W}riter}, considero que
la presentación y calidad del documento merecen la pena.

Por último, quiero agradecer a los profesores y desarrolladores del
servidor la inclusión de distintos clientes en diferentes lenguajes de
programación ya que nos sirven de ejemplo para futuros usos del
interfaz de comunicaciones {I}ce.